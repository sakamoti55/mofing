\RequirePackage{plautopatch}
\RequirePackage[l2tabu, orthodox]{nag}

\documentclass[platex,dvipdfmx]{jlreq}			% for platex
% \documentclass[uplatex,dvipdfmx]{jlreq}		% for uplatex
\usepackage{graphicx}
\usepackage{bxtexlogo}

\title{音素材モーフィング 実験レポート}

\author{学生番号5422021 望月晃星}
\date{\today}
\begin{document}
\maketitle

\begin{abstract}%概要
本レポートでは、2つの音素材同士をモーフィングする実験について述べる。2つの音素材とモーフィングにより作成された音源の関連性や変化を調べる為、特徴的でかつ変化の少ない音源を素材として扱う。
\end{abstract}

\section{はじめに}
音素材をモーフィングするというのは、

\section{はじめに}
\section{はじめに}
\section{はじめに}

\begin{figure}
\centering
\includegraphics[width=70mm]{figures/Sample.png}
\caption{ここにキャプションを挿入します}
\label{fig:model}
\end{figure}

(図\ref{fig:model})
\end{document}